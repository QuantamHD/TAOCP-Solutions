\providecommand{\main}{..}
\documentclass[../../main.tex]{subfiles}

\begin{document}
  \subsubsection{Solutions to Exercises}
    \begin{enumerate}
      \item This should be the minimum number of replacements. Just as a comment this
      is basically emulating a shift register.

        \begin{equation*}
          \begin{split}
            t &\leftarrow a\\
            a &\leftarrow b\\
            b &\leftarrow c\\
            c &\leftarrow d\\
            d &\leftarrow t\\
          \end{split}
        \end{equation*}

        As you can see this will result in $(b,c,d,a)$.

      \item We will define a run of an algorithm as each recursive call to the algorithm in question.

      \begin{theorem}
        If it is not the first run of the \textbf{Algorithm E}, then $m \geq n$.
      \end{theorem}

      \begin{proof}
        We will show with a proof by contradiction that if it is not first run of \textbf{Algorithm E}
        then $m \geq n$. We will label each successive run of the \textbf{Algorithm E} as a natural number
        $q$ starting from 1. We will assume that on a given run $q > 1$ that $m < n$ at step \textbf{E1}.

        \par If $m < n$ then we know that the result of step \textbf{E3} at run $q-1$ resulted in $m < n$. For that
        to have been the case we know that after step \textbf{E1} on run $q-1$, $n$ must be less than $r$. However
        this contradicts the condition that after step \textbf{E1}
        \begin{equation*}
          0 \leq r < n.
        \end{equation*}

        Thus by contradiction we have shown that if is not the first run of \textbf{Algorithm E} then $m \geq n$.
      \end{proof}
    \end{enumerate}

\end{document}
